\documentclass[12pt]{report}
\RequirePackage[l2tabu, orthodox]{nag}
\usepackage[utf8]{inputenc}
\usepackage[russian]{babel}
\usepackage[top=2.3cm, bottom=3.3cm, left=2.5cm, right=2.5cm]{geometry}
\usepackage{amssymb,amsthm,mathtools,microtype,hyperref}

\newtheorem{theorem}{Теорема}
\newtheorem{example}{Пример}

\renewcommand{\thesection}{\arabic{section}}

\delimitershortfall-1sp
\newcommand\abs[1]{\left|#1\right|}

\hypersetup{
    colorlinks=true,
    linkcolor=black,
    filecolor=magenta,      
    urlcolor=black,
    pdftitle={Лекция 1. Числовые ряды},
    bookmarks=true,
}

\begin{document}
\section{Основные понятия}
Пусть $\{a_n\}$ --- произвольная числовая последовательность. Запись этой последовательности в виде
\begin{equation} a_1 + a_2 + \ldots + a_n + \ldots~~~~\text{или}~~~~\sum_{k=1}^{\infty}a_k \end{equation} \label{eq:1}
называют \textbf{числовым рядом}.

Сумма нескольких последовательных членов ряда $S_n = a_1 + \ldots + a_n$ называется \textbf{частичной суммой} этого ряда. Частичные суммы $S_n$ образуют последовательность, называемую \textbf{последовательностью частичных сумм}. Предел последовательности частичных сумм, если он существует, называют \textbf{суммой ряда}. Таким образом, под суммой $S$ ряда \eqref{eq:1} понимается предел
\[ S = \lim_{n \to \infty} \left( \sum_{k=1}^{n} a_k \right).\]

Если указанный предел частичных сумм ряда существует, то говорят, что \textbf{ряд сходится}. В противном случае говорят, что \textbf{ряд расходится}.

Между последовательностями и рядами существует тесная связь. Каждый ряд $\sum a_k$ генерирует специальную последовательность --- последовательность своих частичных сумм $\{ S_n \}$. В то же время любая числовая последовательность $\{ S_n \}$ является последовательностью частичных сумм некоторого ряда, именно, достаточно положить $a_1 = S_1$, $a_n = S_n - S_{n-1}$.

\begin{example}
\[ \sum_{n = 1}^{\infty} \frac{1}{n} \text{ --- гармонический ряд.}\]
\end{example}

\begin{example}
\[ \sum_{n = 0}^{\infty} a \cdot q^n \text{ --- геометрическая прогрессия.} \]

\[S_n = \sum_{i=0}^{n} a \cdot q^i = a \cdot \frac{q^{n+1} - 1}{q - 1}.\]
При $n \to \infty$ имеем два случая:
\begin{enumerate}
\item $\abs{q} < 1$, $\lim\limits_{n \to \infty} q^n = 0$, $S_n \to \dfrac{a}{q - 1}$;
\item $\abs{q} > 1$, $\lim\limits_{n \to \infty} q^{n+1}$ не существует.
\end{enumerate}
Итого: $\sum\limits_{n = 0}^{\infty} a \cdot q ^n$ сходится при $\abs{q} < 1$, расходится при $\abs{q} > 1$.
\end{example}

\begin{theorem}[Необходимый признак сходимости]
\end{theorem}
\end{document}
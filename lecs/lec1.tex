\documentclass[12pt]{report}
\RequirePackage[l2tabu, orthodox]{nag}
\usepackage[utf8]{inputenc}
\usepackage[russian]{babel}
\usepackage[top=2.3cm, bottom=3.3cm, left=2.5cm, right=2.5cm]{geometry}
\usepackage{amssymb,amsthm,mathtools,microtype,hyperref}

\newtheorem{theorem}{Теорема}
\newtheorem{example}{Пример}

\renewcommand{\thesection}{\arabic{section}}

\delimitershortfall-1sp
\newcommand\abs[1]{\left|#1\right|}

\hypersetup{
    colorlinks=true,
    linkcolor=black,
    filecolor=magenta,      
    urlcolor=black,
    pdftitle={Лекция 1. Числовые ряды},
    bookmarks=true,
}

\begin{document}
\section{Основные понятия}
Пусть $\{a_n\}$ --- произвольная числовая последовательность. Запись этой последовательности в виде
\begin{equation} a_1 + a_2 + \ldots + a_n + \ldots~~~~\text{или}~~~~\sum_{k=1}^{\infty}a_k \end{equation} \label{eq:1}
называют \textbf{числовым рядом}.

Сумма нескольких последовательных членов ряда $S_n = a_1 + \ldots + a_n$ называется \textbf{частичной суммой} этого ряда. Частичные суммы $S_n$ образуют последовательность, называемую \textbf{последовательностью частичных сумм}. Предел последовательности частичных сумм, если он существует, называют \textbf{суммой ряда}. Таким образом, под суммой $S$ ряда \eqref{eq:1} понимается предел
\[ S = \lim_{n \to \infty} \left( \sum_{k=1}^{n} a_k \right).\]

Если указанный предел частичных сумм ряда существует, то говорят, что \textbf{ряд сходится}. В противном случае говорят, что \textbf{ряд расходится}.

Между последовательностями и рядами существует тесная связь. Каждый ряд $\sum a_k$ генерирует специальную последовательность --- последовательность своих частичных сумм $\{ S_n \}$. В то же время любая числовая последовательность $\{ S_n \}$ является последовательностью частичных сумм некоторого ряда, именно, достаточно положить $a_1 = S_1$, $a_n = S_n - S_{n-1}$.

\begin{example} \label{ex:1}
\[ \sum_{n = 1}^{\infty} \frac{1}{n} \text{ --- гармонический ряд.}\]
\end{example}

\begin{example} \label{ex:2}
\[ \sum_{n = 0}^{\infty} a \cdot q^n \text{ --- геометрическая прогрессия.} \]

\[S_n = \sum_{i=0}^{n} a \cdot q^i = a \cdot \frac{q^{n+1} - 1}{q - 1}.\]
При $n \to \infty$ имеем два случая:
\begin{enumerate}
\item $\abs{q} < 1$, $\lim\limits_{n \to \infty} q^n = 0$, $S_n \to \dfrac{a}{q - 1}$;
\item $\abs{q} > 1$, $\lim\limits_{n \to \infty} q^{n+1}$ не существует.
\end{enumerate}
Итого: $\sum\limits_{n = 0}^{\infty} a \cdot q ^n$ сходится при $\abs{q} < 1$, расходится при $\abs{q} > 1$.
\end{example}

\begin{theorem}[Необходимый признак сходимости]
Если ряд \eqref{eq:1} сходится, то $\lim\limits_{n \to \infty} a_n = 0.$
\end{theorem}
\begin{proof}
\[ a_n = S_n - S_{n-1} \to S - S = 0.\]
\end{proof}

\begin{example} \label{ex:3}
В случае из примера \ref{ex:2}, когда $q = -1$, $a_n = (-1)^n \cdot a$: суммы ряда не существует (нет предела), ряд расходится.
\end{example}

\begin{example} \label{ex:4}
Рассмотрим пример \ref{ex:1}. В нем
\[ S_{2n} = S_n + \frac{1}{n + 1} + \dots + \frac{1}{2n} > S_n + \underbrace{\frac{1}{2n} + \dots + \frac{1}{2n}}_{\text{n раз}} = S_n + \frac{1}{2}.\]
Рассмотрим $n = 2^p$:
\[S_n > S_{\frac{1}{2}} + \frac{1}{2} > S_{\frac{1}{4}} + \frac{1}{2} > \dots > \frac{p}{2}.\]
Значит, $\{ S_{2^ p}\}$ неограничена, а следовательно, расходится. Тогда и $\{ S_n \}$ расходится. В итоге получаем, что гармонический ряд расходится, при том, что $a_n \to 0$.\\
\end{example}

Рассмотрим ряд
\begin{equation} \label{eq:2}
\sum_{n = 1}^{\infty} a_n = S_{n-1} + R_k.
\end{equation}
$R_k = \sum\limits_{n = k}^{\infty} a_n$ --- \textbf{остаток} ряда \eqref{eq:2}.

\begin{theorem}[Об остатке]
Ряд \eqref{eq:2} сходится $\iff$ сходятся его остатки.
\end{theorem}
\begin{proof}
\[S_n = \sum\limits_{i = 1}^{n}a_i,~S_l' = \sum\limits_{i = k}^{k + l} a_i,~S_{k+l} = S_{k-1} + S_l'.\]\\

''$\implies$``~:
\[ \lim_{k \to \infty}S_{k+l} = S \implies  S_l \to S - S_{k-1}.\]

'' $\Longleftarrow$ ``~:
\[ S_l' \to S' \implies S_{k+l} \to S_{k-1} + S'.\]
\end{proof}

\section{Операции над рядами}

\begin{enumerate}
\item Произведение ряда \eqref{eq:2} на число $\lambda$:
\[ \lambda \sum_{n = 1}^{\infty} a_n = \sum_{n = 1}^{\infty} (\lambda a_n).\]
\item Сумма (разность) двух рядов:
\[ \sum_{i = 1}^{\infty} a_i \pm \sum_{i = 1}^{\infty} b_i = \sum_{i = 1}^{\infty} (a_i \pm b_i).\]
\end{enumerate}

\section{Признаки сравнения}
\subsection{Знакоположительные ряды}
\[ \exists k \in \mathbb{N} : \forall i \geqslant k~a_n \geqslant 0.\]
Аналогично могут быть определены \textbf{знакоотрицательные} ряды:
\[ \exists k \in \mathbb{N} : \forall i \geqslant k~a_n \leqslant 0.\]

\begin{theorem}[1-й признак сравнения]
Пусть $A = \sum\limits_{n = 1}^{\infty} a_n$ и  $B = \sum\limits_{n = 1}^{\infty}b_n$  --- знакоположительные ряды.
$\exists k \in \mathbb{N} : \forall n > k~ a_n \leqslant b_n$, тогда:
\begin{enumerate}
\item Если $B$ сходится, то и $A$ тоже сходится
\item Если $A$ расходится, то и $B$ тоже расходится
\end{enumerate}
\end{theorem}
% Возможно, стоит привести доказательство Четверикова, ибо оно подробнее. Но нужно тогда найти понятный конспект
\begin{proof}
Обозначив через $\{ s_n \}$ последовательность частичных сумм ряда $A$, через $\{ S_n \}$ --- ряда $B$, заключаем, что $s_n \leqslant S_n$, $n = \overline{1, \infty}$. Если ряд $B$ сходится, то последовательность $\{S_n\}$ монотонно возрастает и ограничена. Но тогда и монотонно возрастающая последовательность $\{s_n\}$ ограничена, т.е. ряд $A$ сходится. Второе утверждение является логическим отрицанием первого.
\end{proof}
Обозначение: $A \leqslant B$. Говорят, что ряд $B$ является \textbf{мажорирующим} для ряда $A$.

\begin{theorem}[Предельный признак сравнения]
Пусть для знакоположительных рядов $A = \sum a_n$ и $B = \sum b_n$ существует конечный ненулевой предел
\[ \lim_{n \to \infty} \frac{a_n}{b_n} = c \in \mathbb{R}.\]
Тогда ряды $A$ и $B$ либо оба сходятся, либо оба расходятся.
\end{theorem}
\begin{proof}
Ряды $A$ и $B$ знакоположительны. $\dfrac{a_n}{b_n}$ ограничено, 
\[\exists c_1, c_2 : c_1 \leqslant \frac{a_n}{b_n} c_2 \implies a_n \leqslant c_2 \cdot b_n.\]
Если ряд $B$ сходится, то $A$ \emph{мажорируется} сходящимся рядом $c_2 B$. Поэтому и ряд $A$ сходится. Если же ряд $B$ расходится, и $b_n \leqslant \dfrac{a_n}{c_1}$ $(c_1 > 0)$>, то $\dfrac{1}{c_1}A$ расходится, а значит и $A$ расходится.
\end{proof}

\begin{theorem}[Радикальный признак Коши]
Ряд $A = \sum a_n$, и существует предел 
\[\lim_{n \to \infty} \sqrt[n]{a_n} = q \in \mathbb{R} \implies
\begin{cases}
q < 1 \implies \text{ ряд сходится},\\
q > 1 \implies \text{ ряд расходится}.
\end{cases} \]
\end{theorem}
\begin{proof}
\begin{enumerate}
\item[] % HOLY CRUTCH!
\item[q < 1] Пусть $q < q' < 1 \implies \exists k \in \mathbb{N} : \forall n \geqslant k ~ \sqrt[n]{a_n} \leqslant q'$. То есть $a_n \leqslant (q')^n$ и ряд $A$ мажорируется рядом $\sum\limits_n q^n$. Такой ряд сходится $\implies$ ряд $A$ сходится.
\item[q > 1] $\exists k \in \mathbb{N} : \forall n \geqslant k ~ \sqrt[n]{a_n} > 1$. Тогда $a_n > 1$, $\lim a_n \neq 0$. Необходимый признак не выполняется, ряд расходится.
\end{enumerate}
\end{proof}

\begin{theorem}[Признак Даламбера]
Ряд $A = \sum a_n$ знакопостоянный,
\[ \lim_{n \to \infty} \frac{a_{n+1}}{a_n} = q \implies
\begin{cases}
q < 1 \implies \text{ ряд $A$ сходится},\\
q > 1 \implies \text{ ряд $A$ расходится}.
\end{cases}\]

\end{theorem}
\begin{proof}
:-P
\end{proof}

\begin{theorem}[Интегральный признак Коши]
Пусть функция $f(x)$, определенная на $[0; +\infty)$, неотрицательная и убывающая. Тогда
\[ \int_0^{+\infty} f(x)~dx \text{ и } \sum_{n = 1}^{\infty} f(n) \text{ сходятся и расходятся одновременно}.\]
\end{theorem}

\begin{proof}
В силу монотонности функции $f(x)$ для любого натурального числа $n$ имеем неравенства
\[ f(n+1) \leqslant \int_n^{n+1} f(x)~dx \leqslant f(n).\]
Складывая эти неравенства для различных значений $n$, получаем
\[ \sum_{n = 1}^{N} f(n) = \sum_{n=0}^{N-1} f(n + 1) \leqslant \sum_{n = 0}^{N-1} \int_n^{n+1} f(x)~dx = \int_0^{N} f(x)~dx,\]
\[ \int_1^{N} f(x)~dx = \sum_{n = 1}^{N - 1} \int_n^{n+1} f(x)~dx \leqslant \sum_{n=1}^{N-1} f(n).\]
Если ряд $\sum\limits_{n=1}^{\infty} f(n)$ сходится, то его частичные суммы ограничены. Значит, для некоторого числа $M$ верно неравенство
\[ \int_1^N f(x)~dx \leqslant M.\]
Таким образом, функция $F(x) = \int\limits_{1}^{x} f(x)~dx$ монотонно возрастает, так как подынтегральная функция неотрицательна, причем $F(N) \leqslant M$. Значит, $F(x)$ ограничена и имеет предел при $x \to \infty$. Но существование предела равносильно сходимости несобственного интеграла $\int\limits_1^{\infty} f(x)~dx$.

Если сходится несобственный интеграл $\int\limits_1^{\infty} f(x)~dx$, то функция, определенная равенством $F(x) = \int\limits_{0}^x f(x)~dx$, ограничена. Значит, ограничена последовательность частичных сумм $S_N = \sum\limits_{n = 1}^{N} f(n)$, что равносильно сходимости знакоположительного ряда $\sum\limits_{n = 1}^{\infty} f(n)$.
\end{proof}
\end{document}
